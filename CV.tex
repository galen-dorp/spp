%%%%%%%%%%%%%%%%%%%%%%%%%%%%%%%%%%%%%%%%%%%%%%%%%%%%%%%%%%%%%%%%
%%%%%% CV Template for SPP 2458 "Combinatorial Synergies" %%%%%%
%%%%% Prepared by Galen Dorpalen-Barry, Thomas Kahle, and  %%%%%
%%%%%%%%%%%%%%%%% Laura Voggesberger %%%%%%%%%%%%%%%%%%%%%%%%%%%
%%%%%%%%%%%%%%%%%%%%%%%%%%%%%%%%%%%%%%%%%%%%%%%%%%%%%%%%%%%%%%%%

\documentclass[a4paper,11pt]{article}
\usepackage{amsfonts}
\usepackage{amssymb}
\usepackage{graphicx}
\usepackage{eurosym}
% \usepackage{revnum}
\usepackage{xcolor}
\definecolor{darkblue}{RGB}{0,0,160}
\usepackage{hyperref}
\usepackage{helvet}  %% DFG wünscht sich eigentlich Arial
\usepackage{url}
\usepackage[ngerman]{babel}
\usepackage{tabularx}
\usepackage{bibentry}

\usepackage[utf8]{inputenc}
\usepackage[
text={460pt,640pt},
headheight=18pt,
centering
]{geometry}

\usepackage{fancyhdr}
\pagestyle{fancy}
\cfoot{}		     % Keine Fusszeile
\lhead{{\noindent\myname}}   % Name links
\rhead{{\large Curriculum Vitae \quad\thepage}}% "CV und Seitenzahl rechts

% Zwischenüberschriften
\usepackage[big,compact]{titlesec}	%Erstmal kompakte Standards
\newcommand{\periodafter}[1]{#1.}
\titleformat{\section}
{\large\bfseries}{\thesection}{.5em}{\underline}
\titlespacing{\section}{0pt}{5pt}{5pt}
\titlespacing{\subsection}{0pt}{5pt}{0pt}

\setlength{\parindent}{0pt}

% Allgemeine Daten hier rein:
\def\mytitle{Prof.\ Dr.\ }
\def\myname{MY NAME}
\def\myinstitute{\begin{minipage}[t]{8cm}
    MY UNIVERSITY\\
    Fakultät für Mathematik\\
    MY ADDRESS
    \end{minipage}\vspace{.3ex}}
\def\myphonenr{PHONE}
\def\myemail{MY EMAIL}
\def\myhomepage{\url{https://www.thomas-kahle.de}}
\def\myORCID{\href{https://orcid.org/0000-0002-9271-8436}{ORCID ID}} % don't forget to change the url!
\def\myidentifier{ORCID: \myORCID}
\def\myposition{W2-Professor}


%%%%%%%%%%%%%%%%%%%%%%%%%%%%%%%%%%%%%%%%%%%%%%%%%%%%%%%%%%%%%%%%%%%%%%
\begin{document}%%%%%%%%%%%%%%%%%%%%%%%%%%%%%%%%%%%%%%%%%%%%%%%%%%%%%%
%%%%%%%%%%%%%%%%%%%%%%%%%%%%%%%%%%%%%%%%%%%%%%%%%%%%%%%%%%%%%%%%%%%%%%

\thispagestyle{empty}

\mbox{}
\vspace{-9ex}

\centerline{\Large \bf Curriculum Vitae \myname}
\emph{DFG: Your academic curriculum vitae serves to present your personal profile. It helps DFG’s reviewers and committees in conducting their review and evaluation of your academic achievements and qualifications. The DFG Head Office refers to the curriculum vitae to check you are eligible to apply as well as to see whether there may be any conflicts of interests on the part of reviewers and committee members. All the following details are required unless anything is stated specifically to the contrary.
In terms of providing concrete evidence of proposal-specific or topic-specific qualifications, please refer to the section on own "preliminary work” in the proposal document.
The CV must not exceed four pages. Please make sure to retain the template formatting. In particular, the font should not be smaller than Arial 11 point, with line spacing no less than 1.2. A photograph must not be attached to the curriculum vitae. Please 
name the document CV\_PubList\_$\langle$ person’s last name$\rangle$.
The text in grey or red font provides you with information when preparing your CV. Please remove these texts completely after filling in the CV.
Additional information is available under \url{www.dfg.de/faq_cv}.}
\section*{Persönliche Daten}

\noindent
\begin{tabularx}{\textwidth}{@{}ll}
  \textbf{Titel und Name} & \mytitle \myname \\
  \textbf{Aktuelle Position} & \myposition \\
  \textbf{Aktuelle Institution(en)} & \myinstitute \\
  \textbf{Identifikatoren} & \myidentifier
\end{tabularx}


\section*{Qualifizierung und Werdegang}

\noindent
\begin{tabularx}{\textwidth}{@{}lX}
  \textbf{Studium} & Fach, Zeitraum, Ort, Land \\
  \textbf{Promotion} & Datum, Betreuer:innen, Fach, Einrichtung\\
  \textbf{Stationen} & \emph{DFG: Für den Antrag relevante Tätigkeiten sind chronologisch (die aktuellste am Anfang) mit der Angabe von Zeitraum, Station/Position und Einrichtung zu nennen, wie z. B. Forschungsaufenthalte, Habilitation (Thema/Fach, Betreuende), Tätigkeiten an Hochschulen/außeruniversitären Einrichtungen, klinische Tätigkeit/Tätigkeit in der Versorgung von Patientinnen und Patienten, Erfahrungen und Qualifikationen in der Durchführung von klinischen Studien (Qualifikation als Prüferin, Prüfer oder Prüfungsteam-Mitglied sowie regulatorische und methodische Kenntnisse), Tätigkeit in der Industrieforschung, Tätigkeiten in anderen Berufsbranchen, Unternehmensgründungen, ehrenamtliche Tätigkeiten etc.}
\end{tabularx}

\section*{Ergänzende Angaben zum Werdegang}

\emph{DFG: Hier können Sie freiwillig ergänzende Informationen zu
  Ihrem Werdegang oder einer besonderen persönlichen Situation
  eintragen, sollten Sie den Eindruck haben, dass diese Angaben für
  die angemessene Begutachtung und Bewertung Ihrer wissenschaftlichen
  Leistung relevant sein können. Als solche Besonderheiten oder
  Verzögerungen können beispielsweise Ausfallzeiten aufgrund von
  Kinderbetreuungsaufgaben, Mutterschutz-, Eltern- oder
  Erziehungszeiten, chronischen/langfristigen Erkrankungen, einer
  Behinderung oder besonderen familiären Verpflichtungen, wie der
  Pflege von Angehörigen, sowie Pandemie-bedingten Ausfallzeiten
  berücksichtigt werden. Es können auch zeitliche Verzögerungen im
  wissenschaftlichen Werdegang, z. B. für Personen, die in erster
  Generation eine akademische Karriere anstreben („first generation
  academic“), aufgrund von verschiedenen Pflicht- und
  Freiwilligendiensten, Spracherwerb, Migration oder
  Integrationsphasen, Flucht oder Asylverfahren und Ähnliches genannt
  werden. Bitte nennen Sie keine oder so wenige Daten wie möglich von
  Dritten. }

%%%%%%%%%%%%%%%%%%%%%%%%%%%%%%%%%%%%%%%%%%%%%%%%%%%%%%%%%%%%%%%%%%%%%%
\section*{Engagement im Wissenschaftssystem}%%%%%%%%%%%%%%%%%%%%%%%%%%%%%%%%%%%%%%
%%%%%%%%%%%%%%%%%%%%%%%%%%%%%%%%%%%%%%%%%%%%%%%%%%%%%%%%%%%%%%%%%%%%%%

Freitext, optional
\emph{Hier können Sie Angaben zu weiteren Tätigkeiten im
  Wissenschaftssystem machen. Dazu zählen beispielsweise
  Gremientätigkeiten, Tätigkeiten in der Selbstverwaltung der
  Wissenschaft, die Organisation wissenschaftlicher Veranstaltungen,
  Aktivitäten in der Lehre sowie Tätigkeiten als Mentorin
  bzw. Mentor.}

%%%%%%%%%%%%%%%%%%%%%%%%%%%%%%%%%%%%%%%%%%%%%%%%%%%%%%%%%%%%%%%%%%%%%%
\section*{Betreuung von Forschenden in frühen Karrierephasen}%%%%%%%%%%%%%%%%%%%%%%%%%%%%%%%%%%%%%%
%%%%%%%%%%%%%%%%%%%%%%%%%%%%%%%%%%%%%%%%%%%%%%%%%%%%%%%%%%%%%%%%%%%%%%

Freitext, optional.

\section*{Wissenschaftliche Ergebnisse}

\textbf{A: 10 ausgewählte Publikationen}

\nobibliography{bibliography}
\bibliographystyle{unsrt}

\begin{enumerate}
  \item \bibentry{thomas}
  \item \bibentry{raman}
\end{enumerate}

\textbf{B: 10 sonstige Veröffentlichungen}

\emph{DFG: optional, Freifeld, Preprints, Dinge ohne Peer Review, Patente, Blogposts, ...}

\begin{enumerate}
  \item \bibentry{bernd}
\end{enumerate}

\section*{Anerkennung durch das Wissenschaftssystem}

\emph{DFG: Hier können Sie Angaben zu Auszeichnungen oder Preisen
  machen. Dazu zählen auch Einladungen oder Berufungen in
  herausgehobenen Gremien oder Akademien.  }

\section*{Sonstige Angaben}

\emph{Hier können Sie auf weitere Punkte zur Charakterisierung Ihrer Person als Wissenschaftlerin bzw. Wissenschaftler oder auf andere Aspekte, wie beispielsweise eine Dual-Career-Thematik (die z.  B. die Standortwahl bedingt), hinweisen, die aus Ihrer Sicht relevant für die Begutachtung oder Bewertung des Antrags sind.}

\section*{Datenschutz und Einwilligung in die Verarbeitung optionaler Angaben}
\footnotesize

\noindent
\textbf{Ich willige ausdrücklich in die Verarbeitung der freiwilligen
(optionalen) Angaben, einschließlich „besonderer Kategorien
personenbezogener Daten“ zum Zwecke der Prüfung und Entscheidung über
meinen Antrag durch die DFG ein}. Dies beinhaltet auch die
Weiterleitung meiner Daten an die am Entscheidungsprozess beteiligten
externen Gutachtenden, Gremienmitglieder sowie ggf. ausländische
Partnerorganisationen. Soweit diese Empfängerinnen und Empfänger ihren
Sitz in einem Drittland (außerhalb des Europäischen Wirtschaftraums)
haben, willige ich zusätzlich darin ein, dass diesen Zugriff auf meine
Daten zu oben genannten Zwecken gewährt wird, obwohl unter Umständen
kein mit dem EU-Recht vergleichbares Datenschutzniveau gewährleistet
ist. Daher ist die Einhaltung der Datenschutz-Grundsätze des
Unionsrechts nicht garantiert. Insoweit kann es zu einer Verletzung
meiner Grundrechte und Grundfreiheiten und daraus resultierender
Schäden kommen. Dadurch kann es mir erschwert sein, meine Rechte gemäß
der Datenschutz-Grundverordnung (z. B. Auskunft, Berichtigung,
Löschung, Schadensersatz) geltend zu machen und ggf. mit Hilfe von
Behörden oder gerichtlich durchzusetzen.
%
Meine Einwilligung kann ich jederzeit ganz oder in Teilen – mit
Wirkung für die Zukunft, frei und ohne Angabe von Gründen – gegenüber
der DFG widerrufen (postmaster@dfg.de). Die Rechtmäßigkeit der bis
dahin erfolgten Verarbeitung bleibt davon unberührt. Soweit ich in
diesem CV „besondere Kategorien personenbezogener Daten“ Dritter
übermittele, sichere ich zu, dass die insoweit erforderliche
datenschutzrechtliche Legitimation besteht (z. B. durch eine
Einwilligung).
%
Die Datenschutzhinweise zur Forschungsförderung der DFG, die ich unter
\url{https://www.dfg.de/service/datenschutz} abrufen kann, habe ich
zur Kenntnis genommen und leite diese an solche Personen weiter, deren
Daten die DFG verarbeitet, weil sie in diesem CV aufgeführt sind.

%%%%%%%%%%%%%%%%%%%%%%%%%%%%%%%%%%%%%%%%%%%%%%%%%%%%%%%%%%%%%%%%%%%%%%
\end{document}%%%%%%%%%%%%%%%%%%%%%%%%%%%%%%%%%%%%%%%%%%%%%%%%%%%%%%%%
%%%%%%%%%%%%%%%%%%%%%%%%%%%%%%%%%%%%%%%%%%%%%%%%%%%%%%%%%%%%%%%%%%%%%%