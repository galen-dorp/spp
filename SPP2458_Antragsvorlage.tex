%%%%%%%%%%%%%%%%%%%%%%%%%%%%%%%%%%%%%%%%%%%%%%%%%%%%%%%%%%%%%%%%
%%%%%%%%%%%%%%%%%% Projektantrag Template for %%%%%%%%%%%%%%%%%%
%%%%%%%%%%%%%% SPP 2458 "Combinatorial Synergies" %%%%%%%%%%%%%%
%%%% Prepared by Galen Dorpalen-Barry and %%%%
%%%%%%%%%%%%%%%%% Laura Voggesberger %%%%%%%%%%%%%%%%%%%%%%%%%%%
%%%%%%%%%%%%%%%%%%%%%%%%%%%%%%%%%%%%%%%%%%%%%%%%%%%%%%%%%%%%%%%%

\documentclass[a4paper,11pt]{article} %LV: changed font size to 11pt, as the DFG says that it should be at least size 11pt

\renewcommand{\baselinestretch}{1.2} %LV: the DFG asks for line spacing of at least 1.2
\usepackage{helvet}  %LV: changed font, but note that the DFG would like to have Arial
\usepackage{multibib}
\usepackage{bibentry}
\newcites{all}{\vspace{-10pt}}
\newcites{mine}{\vspace{-10pt}}

\usepackage{a4wide}
\usepackage[utf8]{inputenc}
\usepackage{amssymb}
\usepackage{amsmath}
\usepackage[small]{titlesec}
\usepackage{enumerate}
\usepackage{graphicx}
\usepackage{url}

\def\fix#1{{\bf FIXME:} #1 \\}
\def\Dfn#1{\emph{#1}}
\def\part#1{\noindent{\bf #1}\\[5pt]}

%Note that for the SPP we have the following: 
% Each proposal must be accompanied by a description of how the project is integral to the
%Priority Programme, both in terms of subject matter and organisation. This includes a descrip-
%tion of the cooperation with others participating within the Priority Programme. The envisaged
%realisation of the project in cooperation with other applicants may be demonstrated in particular
%by the joint training of early career researchers, or the use of methods by multiple projects as
%part of a network.
%All applicants involved in submitting a proposal within an established Priority Programme are
%obliged to promptly provide the overall coordinator with all of the information necessary for
%drawing up the interim reports and the final report for the Priority Programme.
\begin{document}

\title{{\small Projektantrag im Rahmen einer Sachbeihilfe}\\{\bf \LARGE PROJECT TITLE}}
\author{MY NAME}
\date{\today}
\maketitle

\noindent This is a revised and shortened version of a first-time proposal submitted to the DFG [...] in 2022.

\section{State of the art and own previous work}
%quote from the DFG: "For new proposals please explain briefly and precisely the state of the art in your field in
%its direct relationship to your project. This description should make clear in which context
%you situate your own research and in what areas you intend to make a unique, innova-
%tive, promising contribution. Indicate the current state of your preliminary work. This de-
%scription must be concise and understandable without referring to additional literature.
%For renewal proposals, please report on your previous work. This report should also be
%understandable without referring to additional literature."
\paragraph{...}

\subsection{Project-related list of publications}

\subsubsection{Peer-reviewed publications}

\nobibliography{bibliography}
\bibliographystyle{unsrt}

\begin{itemize}
  % \setlength{\itemindent}{18px}
  \item \bibentry{martina}
\end{itemize}

\subsubsection{Other publications}

\begin{itemize}
  \item \bibentry{christian}
\end{itemize}

\section{Objectives and work schedule}

\subsection{Anticipated total duration}

The anticipated total duration is 36 months in which I plan to achieve the following objectives.

\subsection{Objectives}

\subsection{Work schedule}\label{sec:work schedule}
%DFG: "Please give a detailed account of the steps planned during the proposed funding
%period. (For experimental projects, a schedule detailing all planned experiments should
%be provided.)
%The quality of the work programme is critical to the success of a funding proposal. The
%tasks to be performed within the work programme should correspond to the funds re-
%quested. The work programme should therefore indicate and justify what types of funding
%will be needed and how the funds will be used, providing details on the individual items
%requested where applicable.
%Please provide a detailed description of the methods that you plan to use in the project:
%What methods are already available? What methods need to be developed? What as-
%sistance is needed from outside your own group/institute?
%
%Concepts and starting points for quality-promoting measures that specifically contribute
%to the validity or plausibility of your research results are welcome here. For more in-depth
%and subject-specific recommendations, see the “Research Integrity” portal."

\subsubsection{Providing a central hub within Coxeter-Catalan combinatorics}
\label{sec:centralhub}

%%%%%%%%%%%%%%%%%%%%%%%%%%%%%%%%%%%%%%%%%%%%%%%%%

\subsection{Data handling}

All accumulated data from computer experiments will be published within [...]

%DFG:"If your project uses, generates and/or processes data, use this section to record key
%information on the handling of this data (and any underlying objects). Please ensure your
%descriptions substantively follow the points in the corresponding questionnaire
%(www.dfg.de/forschungsdaten/checkliste) and use the checklist to address the following
%aspects in particular:
% * Characteristics and scope of data
% * Documentation and data quality
% * Storage and technical archiving
% * Legal obligations and conditions
% * Enabling subsequent reuse and long-term accessibility
% * Responsibilities and resources
%Please also describe how the institutions involved in the project will contribute to data
%and information management.
%If you have already provided more detailed information on the handling of research data
%in an explanation as part of your preliminary work, work programme or elsewhere, you
%may refer to those descriptions and limit yourself to supplementary information at this
%point.
%Should your project not use or generate data to a relevant extent, please explicitly state
%this to be the case.
%Please also note that you can apply for funding to cover project costs associated with
%the effort involved in collecting research data.
%For further information on this topic, see:
%www.dfg.de/en/research_funding/principles_dfg_funding/research_data"

\subsection{Relevance of sex, gender and/or diversity}
%DFG: Where applicable, please describe whether and to what extent the sex and/or gender
% * of researchers
% * of persons under study
% * of individuals affected by the implementation of research results
% * of animals under study
% * with regard to samples taken from humans or animals
% * in other respects
%is relevant to the research project (methods, work programme, objectives, etc.).
%Where applicable, please also describe whether and to what extent diversity in terms of,
%for example, the state of health, ethnic background or culture of
% * researchers
% * persons under study
% * individuals affected by the implementation of research results
% * or diversity in other respects
%may be significant for the research project (methods, work programme, objectives, etc.).
%Please explain to what extent these or similar considerations may also be relevant to
%animals under study or samples taken from humans or animals.
%Additional information is available at
%www.dfg.de/diversity_dimensions"
\section{Bibliography}

{\fontsize{9}{11}\selectfont %LV: Here, a smaller font size is allowed (however no smaller than 9pt)
%DFG:"This list should only contain those works that you cited in sections 1 and 2.
%For both new proposals and renewal proposals, you can refer to your own works and
%those of others; there is no limit to the total number of publications listed. Works which
%are not in the public domain are not considered publications and cannot be cited. An
%exception is made for papers that have already been accepted for publication, in which
%case the manuscript and the editor’s confirmation of acceptance must be enclosed.
%A maximum of ten of your own publications that are most relevant to the project can
%be highlighted in bold or some other way. Even if there are several applicants, the max-
%imum of ten highlighted works may not be exceeded.
}

%the following sections must not exceed 8 pages in total
\section{Supplementary information on the research context}

\subsection{Ethical and/or legal aspects of the project}

\subsubsection{Employment status information}
%DFG: For each applicant, state the last name, first name, and employment status (including
%duration of contract and funding body, if on a fixed-term contract).

The applicant [XXX] is currently employed at the
\begin{itemize}
  \item ...
\end{itemize}
Their contract will end {\bf END DATE}.

\subsection{First-time proposal data}
%Only if applicable: Last name, first name of first-time applicant.

\section{Composition of the project group}
%List only those individuals who will work on the project but will not be paid out of the
%project funds. State each person’s name, academic title, employment status, and type of
%funding.
\subsection{Cooperation with other researchers}

\subsubsection{Project cooperations}

\begin{itemize}
  \item ...
\end{itemize}

\subsubsection{Collaborators in the past 3 years}

\begin{enumerate}[(a)]
  \item Collaborators on topics related to the project:
  \begin{itemize}
    \item ...
  \end{itemize}
  \item Coauthors within the past 3 years:
  \begin{itemize}
    \item ...
  \end{itemize}
\end{enumerate}

\subsection{Scientific equipment}

\subsection{Other submissions}

\subsection{Other information}

\section{Signature}

\hspace*{75pt} MY NAME \hfill MY CITY, DATE (MONTH YEAR)

\section{List of attachments}

\begin{enumerate}
  \item Curriculum Vitae
  \item PhD certificate
  \item Declarations by the hosting institution will be sent directly to the DFG
\end{enumerate}

\vfill

\end{document}
